\documentclass[12pt]{report}
\usepackage[utf8]{inputenc}
\usepackage[english, russian]{babel}
\usepackage{listings}
\usepackage{graphicx}
\usepackage{float}
\graphicspath{{imgs/}}
\usepackage{amsmath,amsfonts,amssymb,amsthm,mathtools} 
\usepackage{pgfplots}
\usepackage{filecontents}
\usepackage{indentfirst}
\usepackage{eucal}
\usepackage{enumitem}
\frenchspacing

\usepackage{indentfirst} % Красная строка

\usetikzlibrary{datavisualization}
\usetikzlibrary{datavisualization.formats.functions}

\usepackage{amsmath}
\usepackage{fixltx2e}
\usepackage{caption}


\definecolor{bluekeywords}{rgb}{0,0,1}
\definecolor{greencomments}{rgb}{0,0.5,0}
\definecolor{redstrings}{rgb}{0.64,0.08,0.08}
\definecolor{xmlcomments}{rgb}{0.5,0.5,0.5}
\definecolor{types}{rgb}{0.17,0.57,0.68}

\usepackage{listings}
\lstset{language=[Sharp]C,
	captionpos=t,
	numbers=left, %Nummerierung
	numberstyle=\small, % kleine Zeilennummern
	frame=single, % Oberhalb und unterhalb des Listings ist eine Linie
	stepnumber=1,                   
	numbersep=5pt,                
	showspaces=false,
	tabsize=2,
	showtabs=false,
	breaklines=true,
	showstringspaces=false,
	breakatwhitespace=true,
	escapeinside={(*@}{@*)},
	commentstyle=\color{greencomments},
	morekeywords={partial, var, value, get, set},
	keywordstyle=\color{bluekeywords},
	stringstyle=\color{redstrings},
	basicstyle=\ttfamily\small,
}

\usepackage[left=2cm,right=2cm, top=2cm,bottom=2cm,bindingoffset=0cm]{geometry}
% Для измененных титулов глав:
\usepackage{titlesec, blindtext, color} % подключаем нужные пакеты
\definecolor{gray75}{gray}{0.75} % определяем цвет
\newcommand{\hsp}{\hspace{20pt}} % длина линии в 20pt
% titleformat определяет стиль
\titleformat{\chapter}[hang]{\Huge\bfseries}{\thechapter\hsp\textcolor{gray75}{|}\hsp}{0pt}{\Huge\bfseries}

\usepackage{array}
\newcommand{\head}[2]{\multicolumn{1}{>{\centering\arraybackslash}p{#1}}{#2}}

% plot
\usepackage{pgfplots}
\usepackage{filecontents}
\usetikzlibrary{datavisualization}
\usetikzlibrary{datavisualization.formats.functions}

\begin{document}
%\def\chaptername{} % убирает "Глава"
\thispagestyle{empty}
\begin{titlepage}
	\noindent \begin{minipage}{0.15\textwidth}
		\includegraphics[width=\linewidth]{b_logo}
	\end{minipage}
	\noindent\begin{minipage}{0.9\textwidth}\centering
		\textbf{Министерство науки и высшего образования Российской Федерации}\\
		\textbf{Федеральное государственное бюджетное образовательное учреждение высшего образования}\\
		\textbf{~~~«Московский государственный технический университет имени Н.Э.~Баумана}\\
		\textbf{(национальный исследовательский университет)»}\\
		\textbf{(МГТУ им. Н.Э.~Баумана)}
	\end{minipage}
	
	\noindent\rule{18cm}{3pt}
	\newline\newline
	\noindent ФАКУЛЬТЕТ $\underline{\text{«Информатика и системы управления»}}$ \newline\newline
	\noindent КАФЕДРА $\underline{\text{«Программное обеспечение ЭВМ и информационные технологии»}}$\newline\newline\newline\newline\newline\newline\newline\newline\newline\newline\newline
	
	
	\begin{center}
		\noindent\begin{minipage}{1.3\textwidth}\centering
			\Large\textbf{  Отчет по лабораторной работе №7}\newline
			\textbf{по дисциплине \newline "Функциональное и логическое программирование"}\newline\newline
		\end{minipage}
	\end{center}
	
	\noindent\textbf{Тема} $\underline{\text{Рекурсивные функции}}$\newline\newline
	\noindent\textbf{Студент} $\underline{\text{Малышев И. А.}}$\newline\newline
	\noindent\textbf{Группа} $\underline{\text{ИУ7-61Б}}$\newline\newline
	\noindent\textbf{Оценка (баллы)} $\underline{\text{~~~~~~~~~~~~~~~~~~~~~~~~~~~}}$\newline\newline
	\noindent\textbf{Преподаватель: } $\underline{\text{Толпинская Н. Б.}}$\newline\newline\newline
	
	\begin{center}
		\vfill
		Москва~---~\the\year
		~г.
	\end{center}
\end{titlepage}


\setcounter{page}{2}

\chapter*{Практические задания}

\subsection*{1. Написать хвостовую рекурсивную функцию my-reverse, которая развернет верхний
	уровень своего списка-аргумента lst.}

\begin{lstlisting}[label=6xd, caption=Решение задания №1, language=lisp]
(defun my-reverse (lst)
	(my-rev lst ()))
	
(defun my-rev (lst acc)
	(cond ((null lst) acc)
		  (t (my-rev (cdr lst) (cons (car lst) acc)))))

\end{lstlisting}

\subsection*{2. Написать функцию, которая возвращает первый элемент списка-аргумента, который сам
	является непустым списком.}

\begin{lstlisting}[label=6xd, caption=Решение задания №2, language=lisp]
(defun find-first-lst (lst) 
	(cond ((null lst) nil)
		  ((listp (car lst)) (car lst)) 
		  (t (find-first-lst (cdr lst)))))

\end{lstlisting}

\subsection*{3. Написать функцию, которая выбирает из заданного списка только те числа, которые
	больше 1 и меньше 10.}

\begin{lstlisting}[label=6xd, caption=Решение задания №3, language=lisp]


\end{lstlisting}

\subsection*{4. Напишите рекурсивную функцию, которая умножает на заданное число-аргумент все
	числа
	из заданного списка-аргумента, когда}

a) все элементы списка --- числа,

6) элементы списка -- любые объекты.

\begin{lstlisting}[label=6xd, caption=Решение задания №4, language=lisp]


\end{lstlisting}

\subsection*{5. Напишите функцию, select-between, которая из списка-аргумента, содержащего только
	числа, выбирает только те, которые расположены между двумя указанными границамиаргументами и возвращает их в виде списка (упорядоченного по возрастанию списка чисел
	(+ 2 балла)).}

\begin{lstlisting}[label=6xd, caption=Решение задания №5, language=lisp]

	
\end{lstlisting}

\subsection*{6. Написать рекурсивную версию (с именем rec-add) вычисления суммы чисел заданного
	списка:}

а) одноуровнего смешанного,

б) структурированного.

\begin{lstlisting}[label=6xd, caption=Решение задания №6, language=lisp]


\end{lstlisting}

\subsection*{7. Написать рекурсивную версию с именем recnth функции nth.}

\begin{lstlisting}[label=6xd, caption=Решение задания №7, language=lisp]	

	
\end{lstlisting}

\subsection*{8. Написать рекурсивную функцию allodd, которая возвращает t когда все элементы списка
	нечетные.}

\begin{lstlisting}[label=6xd, caption=Решение задания №8, language=lisp]
	
	
\end{lstlisting}

\subsection*{9. Написать рекурсивную функцию, которая возвращает первое нечетное число из списка
	(структурированного), возможно создавая некоторые вспомогательные функции.}

\begin{lstlisting}[label=6xd, caption=Решение задания №9, language=lisp]


\end{lstlisting}

\subsection*{10. Используя cons-дополняемую рекурсию с одним тестом завершения,
	написать функцию которая получает как аргумент список чисел, а возвращает список
	квадратов этих чисел в том же порядке.}

\begin{lstlisting}[label=6xd, caption=Решение задания №10, language=lisp]
	
	
\end{lstlisting}


\end{document}