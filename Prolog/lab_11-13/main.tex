\documentclass[12pt]{report}
\usepackage[utf8]{inputenc}
\usepackage[english, russian]{babel}
\usepackage{listings}
\usepackage{graphicx}
\usepackage{float}
\graphicspath{{imgs/}}
\usepackage{amsmath,amsfonts,amssymb,amsthm,mathtools} 
\usepackage{pgfplots}
\usepackage{filecontents}
\usepackage{indentfirst}
\usepackage{eucal}
\usepackage{enumitem}
\frenchspacing

\usepackage{indentfirst} % Красная строка

\usetikzlibrary{datavisualization}
\usetikzlibrary{datavisualization.formats.functions}

\usepackage{amsmath}
\usepackage{fixltx2e}
\usepackage{caption}


\definecolor{bluekeywords}{rgb}{0,0,1}
\definecolor{greencomments}{rgb}{0,0.5,0}
\definecolor{redstrings}{rgb}{0.64,0.08,0.08}
\definecolor{xmlcomments}{rgb}{0.5,0.5,0.5}
\definecolor{types}{rgb}{0.17,0.57,0.68}

\usepackage{listings}
\lstset{language=[Sharp]C,
	captionpos=t,
	numbers=left, %Nummerierung
	numberstyle=\small, % kleine Zeilennummern
	frame=single, % Oberhalb und unterhalb des Listings ist eine Linie
	stepnumber=1,                   
	numbersep=5pt,                
	showspaces=false,
	tabsize=2,
	showtabs=false,
	breaklines=true,
	showstringspaces=false,
	breakatwhitespace=true,
	escapeinside={(*@}{@*)},
	commentstyle=\color{greencomments},
	morekeywords={partial, var, value, get, set},
	keywordstyle=\color{bluekeywords},
	stringstyle=\color{redstrings},
	basicstyle=\ttfamily\small,
}

\usepackage[left=2cm,right=2cm, top=2cm,bottom=2cm,bindingoffset=0cm]{geometry}
% Для измененных титулов глав:
\usepackage{titlesec, blindtext, color} % подключаем нужные пакеты
\definecolor{gray75}{gray}{0.75} % определяем цвет
\newcommand{\hsp}{\hspace{20pt}} % длина линии в 20pt
% titleformat определяет стиль
\titleformat{\chapter}[hang]{\Huge\bfseries}{\thechapter\hsp\textcolor{gray75}{|}\hsp}{0pt}{\Huge\bfseries}

\usepackage{array}
\newcommand{\head}[2]{\multicolumn{1}{>{\centering\arraybackslash}p{#1}}{#2}}

% plot
\usepackage{pgfplots}
\usepackage{filecontents}
\usetikzlibrary{datavisualization}
\usetikzlibrary{datavisualization.formats.functions}

\begin{document}
	%\def\chaptername{} % убирает "Глава"
	\thispagestyle{empty}
	\begin{titlepage}
		\noindent \begin{minipage}{0.15\textwidth}
			\includegraphics[width=\linewidth]{b_logo}
		\end{minipage}
		\noindent\begin{minipage}{0.9\textwidth}\centering
			\textbf{Министерство науки и высшего образования Российской Федерации}\\
			\textbf{Федеральное государственное бюджетное образовательное учреждение высшего образования}\\
			\textbf{~~~«Московский государственный технический университет имени Н.Э.~Баумана}\\
			\textbf{(национальный исследовательский университет)»}\\
			\textbf{(МГТУ им. Н.Э.~Баумана)}
		\end{minipage}
		
		\noindent\rule{18cm}{3pt}
		\newline\newline
		\noindent ФАКУЛЬТЕТ $\underline{\text{«Информатика и системы управления»}}$ \newline\newline
		\noindent КАФЕДРА $\underline{\text{«Программное обеспечение ЭВМ и информационные технологии»}}$\newline\newline\newline\newline\newline\newline\newline\newline\newline\newline\newline
		
		
		\begin{center}
			\noindent\begin{minipage}{1.3\textwidth}\centering
				\Large\textbf{  Отчет по лабораторной работе №11-13}\newline
				\textbf{по дисциплине \newline "Функциональное и логическое программирование"}\newline\newline
			\end{minipage}
		\end{center}
		
		\noindent\textbf{Тема} $\underline{\text{Среда Visual Prolog. Структура программы. Работа программы}}$\newline\newline
		\noindent\textbf{Студент} $\underline{\text{Малышев И. А.}}$\newline\newline
		\noindent\textbf{Группа} $\underline{\text{ИУ7-61Б}}$\newline\newline
		\noindent\textbf{Оценка (баллы)} $\underline{\text{~~~~~~~~~~~~~~~~~~~~~~~~~~~}}$\newline\newline
		\noindent\textbf{Преподаватель: } $\underline{\text{Толпинская Н. Б.}}$\newline\newline\newline
		
		\begin{center}
			\vfill
			Москва~---~\the\year
			~г.
		\end{center}
	\end{titlepage}
	
	
	\setcounter{page}{2}

\chapter*{Лабораторная работа №11}
\section*{Задание}

Разработать свою программу -- <<Телефонный справочник>>. Протестировать работу программы.

\section*{Решение}
\begin{lstlisting}
domains
	name = string
	phone = string
	city = string
	street = string
	home = integer
	
predicates
	record(name, phone, city, street, home)
	
clauses
	record("Ivan", "+79631416412", "Moscow", "Frunzenskaya", 15).
	record("Peter", "+79123456789", "Saint Peterburg", "Nevskaya", 27).
	record("Sergey", "+79335436781", "Moscow", "Obychnaya", 5).
	record("Marya", "+79027531212", "Moscow", "Yartsevskaya", 13).
	record("Michail", "+79635432121", "Tver", "Chekhovskaya", 33).
	
goal
	record(Name, Phone, "Moscow", Street, Home).
\end{lstlisting}


\chapter*{Лабораторная работа №11(2)}
\section*{Задание}

Составить программу -- базу знаний, с помощью которой можно определить, например, множество студентов, обучающихся в одном ВУЗе и их телефоны. Студент может одновременно обучаться в нескольких ВУЗах. Привести примеры возможных вариантов вопросов и варианты ответов (не менее 3-х). Описать порядок формирования вариантов ответа.

Исходную базу знаний сформировать с помощью только фактов.

*Исходную базу знаний сформировать, используя правила.

**Разработать свою базу знаний (содержание произвольно).

\section*{Решение}
\begin{lstlisting}
domains
	name = string
	surname = string
	phone = string
	university = string
	group = string
	course = integer

predicates
	student(name, surname, phone, university, group, course)

clauses
	student("Ivan", "Malyshev", "+78005553535", "BMSTU", "IU7-61B", 3).
	student("Nikita", "Shatskiy", "+79996663232", "BMSTU", "IU7-61B", 3).
	student("Ivan", "Ivanov", "+79543432255", "MIPT", "LFI-14", 1).
	student("Sergey", "Simonov", "+79342221314", "MAI", "ACS-32", 2).
	student("Ivan", "Malyshev", "+78005553535", "MIPT", "FIVT-5", 5).
	student("Petr", "Grinev", "+79456784433", "MSU", "VMK-4", 4).
	student("Maria", "Bublikova", "+79554327788", University, Group, Course) :- student(\_, "Ivanov", \_, University, Group, Course).

goal
	student("Ivan", Surname, Phone, University, Group, Course).
	student(Name, Surname, Phone, "BMSTU", \_, 3).
	student("Ivan", \_, \_, \_, "IU7-61B", \_).
\end{lstlisting}

С помощью первого вопроса получаются все студенты с именем Иван. Происходит проход сверху вниз по всем фактам предиката \emph{student(name, surname, phone, university, group, course)} и осуществляется унификация с \emph{student(''Ivan'', Surname, Phone, University, Group, Course)}. Унификацию успешно проходит три факта: \emph{student(''Ivan'', ''Malyshev'', ''+78005553535'', ''BMSTU'', ''IU7-61B'', 3)}, \emph{student(''Ivan'', ''Ivanov'', ''+79543432255'', ''MIPT'', ''LFI-14'', 1)} и \emph{student(''Ivan'', ''Malyshev'', ''+78005553535'', ''MIPT'', ''FIVT-5'', 5)}.\\

С помощью второго вопроса получаются все студенты МГТУ, которые обучаются на 3 курсе.  Происходит проход по всем фактам предиката \emph{student(name, surname, phone, university, group, course)} и осуществляется унификация с \emph{student(Name, Surname, Phone, ''BMSTU'', \_, 3)}.  Успешно унификацию проходят факты \emph{student(''Ivan'', ''Malyshev'', ''+78005553535'', ''BMSTU'', ''IU7-61B'', 3)} и \emph{student(''Nikita", ''Shatskiy'', ''+79996663232'', ''BMSTU'', ''IU7-61B'', 3)}.\\

С помощью третьего вопроса получаются все Иваны, которые обучаются в группе ИУ7-61Б.  Происходит проход по всем фактам предиката \emph{student(name, surname, phone, university, group, course)} и осуществляется унификация с \emph{student(''Ivan'', \_, \_, \_, ''IU7-61B'', \_)}.  Успешно унификацию проходит \emph{student(''Ivan'', ''Malyshev'', ''+78005553535'', ''BMSTU'', ''IU7-61B'', 3)}.\\


\end{document}