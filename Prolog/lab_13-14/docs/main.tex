\documentclass[12pt]{report}
\usepackage[utf8]{inputenc}
\usepackage[english, russian]{babel}
\usepackage{listings}
\usepackage{graphicx}
\usepackage{float}
\graphicspath{{imgs/}}
\usepackage{amsmath,amsfonts,amssymb,amsthm,mathtools} 
\usepackage{pgfplots}
\usepackage{filecontents}
\usepackage{indentfirst}
\usepackage{eucal}
\usepackage{enumitem}
\frenchspacing

\usepackage{indentfirst} % Красная строка

\usetikzlibrary{datavisualization}
\usetikzlibrary{datavisualization.formats.functions}

\usepackage{amsmath}
\usepackage{fixltx2e}
\usepackage{caption}


\definecolor{bluekeywords}{rgb}{0,0,1}
\definecolor{greencomments}{rgb}{0,0.5,0}
\definecolor{redstrings}{rgb}{0.64,0.08,0.08}
\definecolor{xmlcomments}{rgb}{0.5,0.5,0.5}
\definecolor{types}{rgb}{0.17,0.57,0.68}

\usepackage{listings}
\lstset{language=[Sharp]C,
	captionpos=t,
	numbers=left, %Nummerierung
	numberstyle=\small, % kleine Zeilennummern
	frame=single, % Oberhalb und unterhalb des Listings ist eine Linie
	stepnumber=1,                   
	numbersep=5pt,                
	showspaces=false,
	tabsize=2,
	showtabs=false,
	breaklines=true,
	showstringspaces=false,
	breakatwhitespace=true,
	escapeinside={(*@}{@*)},
	commentstyle=\color{greencomments},
	morekeywords={partial, var, value, get, set},
	keywordstyle=\color{bluekeywords},
	stringstyle=\color{redstrings},
	basicstyle=\ttfamily\small,
}

\usepackage[left=1cm,right=1cm, top=1cm,bottom=2cm,bindingoffset=0cm]{geometry}
% Для измененных титулов глав:
\usepackage{titlesec, blindtext, color} % подключаем нужные пакеты
\definecolor{gray75}{gray}{0.75} % определяем цвет
\newcommand{\hsp}{\hspace{20pt}} % длина линии в 20pt
% titleformat определяет стиль
\titleformat{\chapter}[hang]{\Huge\bfseries}{\thechapter\hsp\textcolor{gray75}{|}\hsp}{0pt}{\Huge\bfseries}

\usepackage{array}
\newcommand{\head}[2]{\multicolumn{1}{>{\centering\arraybackslash}p{#1}}{#2}}

% plot
\usepackage{pgfplots}
\usepackage{filecontents}
\usetikzlibrary{datavisualization}
\usetikzlibrary{datavisualization.formats.functions}

\begin{document}
	%\def\chaptername{} % убирает "Глава"
	\thispagestyle{empty}
	\begin{titlepage}
		\noindent \begin{minipage}{0.15\textwidth}
			\includegraphics[width=\linewidth]{b_logo}
		\end{minipage}
		\noindent\begin{minipage}{0.9\textwidth}\centering
			\textbf{Министерство науки и высшего образования Российской Федерации}\\
			\textbf{Федеральное государственное бюджетное образовательное учреждение высшего образования}\\
			\textbf{~~~«Московский государственный технический университет имени Н.Э.~Баумана}\\
			\textbf{(национальный исследовательский университет)»}\\
			\textbf{(МГТУ им. Н.Э.~Баумана)}
		\end{minipage}
		
		\noindent\rule{18cm}{3pt}
		\newline\newline
		\noindent ФАКУЛЬТЕТ $\underline{\text{«Информатика и системы управления»}}$ \newline\newline
		\noindent КАФЕДРА $\underline{\text{«Программное обеспечение ЭВМ и информационные технологии»}}$\newline\newline\newline\newline\newline\newline\newline\newline\newline\newline\newline
		
		
		\begin{center}
			\noindent\begin{minipage}{1.3\textwidth}\centering
				\Large\textbf{  Отчет по лабораторной работе №13-14}\newline
				\textbf{по дисциплине \newline "Функциональное и логическое программирование"}\newline\newline
			\end{minipage}
		\end{center}
		
		\noindent\textbf{Тема} $\underline{\text{Структура программы на Prolog и использование правил}}$\newline\newline
		\noindent\textbf{Студент} $\underline{\text{Малышев И. А.}}$\newline\newline
		\noindent\textbf{Группа} $\underline{\text{ИУ7-61Б}}$\newline\newline
		\noindent\textbf{Оценка (баллы)} $\underline{\text{~~~~~~~~~~~~~~~~~~~~~~~~~~~}}$\newline\newline
		\noindent\textbf{Преподаватель: } $\underline{\text{Толпинская Н. Б.}}$\newline\newline\newline
		
		\begin{center}
			\vfill
			Москва~---~\the\year
			~г.
		\end{center}
	\end{titlepage}
	
	
	\setcounter{page}{2}

\chapter*{Лабораторная работа №13}
\section*{Задание}

Создать базу знаний «Собственники», дополнив базу знаний, хранящую знания (лаб. 13):

\begin{itemize}
	\item <<Телефонный справочник>>: Фамилия, №тел, Адрес – структура (Город, Улица, №дома, №кв),
	\item <<Автомобили>>: Фамилия\_владельца, Марка, Цвет, Стоимость, и др.,
	\item <<Вкладчики банков>>: Фамилия, Банк, счет, сумма, др.,
\end{itemize}

знаниями о дополнительной собственности владельца. Преобразовать знания об автомобиле к форме знаний о собственности.

Вид собственности (кроме автомобиля):

\begin{itemize}
	\item Строение, стоимость и другие его характеристики.
	\item Участок, стоимость и другие его характеристики.
	\item Водный\_транспорт, стоимость и другие его характеристики.
\end{itemize}

Описать и использовать вариантный домен: \textbf{Собственность}. Владелец может иметь, но только один объект каждого вида собственности (это касается и автомобиля), или не иметь некоторых видов собственности. 

Используя конъюнктивное правило и разные формы задания одного вопроса (пояснять для какого №задания – какой вопрос), обеспечить возможность поиска:

\begin{enumerate}
	\item Названий всех объектов собственности заданного субъекта.
	\item Названий и стоимости всех объектов собственности заданного субъекта.
	\item Разработать правило, позволяющее найти суммарную стоимость всех объектов собственности заданного субъекта.
\end{enumerate}

Для 2-го пункта и одной фамилии составить таблицу, отражающую конкретный порядок работы системы, с объяснениями порядка работы и особенностей использования доменов  (указать конкретные Т1 и Т2 и полную подстановку на каждом шаге).

\newpage
\section*{Решение}

\begin{lstlisting}
domains
	surname, phone, city, street, color, bank, name = string
	home, flat, cost, account, summ, total = integer
	address = address(city, street, home, flat)
	
	own = building(name, cost);
			region(name, cost);
			car(name, color, cost);
			water_transport(name, color, cost)

predicates
	phone_book(surname, phone, address)
	deposit(surname, bank, account, summ)
	owner(surname, own)
	
	all_owns(surname, name)
	all_owns_cost(surname, name, cost)
	
	all_owns_cost_type(surname, symbol, cost)
	total_cost(surname, total)

clauses
	phone_book("Malyshev", "+78005553535", address("Moscow", "Obychnaya", 11, 2)).
	phone_book("Shatskiy", "+71231421433", address("Saint Peterburg", "Olenevaya", 12, 4)).
	phone_book("Voronin", "+71454663765", address("Saratov", "Bychkovaya", 12, 11)).
	phone_book("Gribochkov", "+71531432289", address("Tver", "Tomatnaya", 12, 7)).
	phone_book("Sazonov", "+71766543721", address("Moscow", "Marmeladnaya", 13, 6)).
	phone_book("Tsetochkin", "+71728332062", address("Tver", "Kabachkovaya", 16, 1)).
	
	owner("Shatskiy", car("Suzuki", "red", 10000000)).
	owner("Gribochkov", car("BMW", "yellow", 15000000)).
	owner("Malyshev", car("Highwayman", "gray", 100000000)).
	owner("Voronin", car("Volga", "black", 20000000)).
	owner("Malyshev", region("Hell", 666666666)).
	owner("Tsetochkin", building("Big garden", 550000)).
	owner("Gribochkov", water_transport("Boatalia", "pink", 55555555)).
	
	deposit("Sazonov", "Sber", 145464235, 1000).
	deposit("Shatskiy", "Tinkoff", 585642576, 20000).
	deposit("Voronin", "Raif", 346536624, 100000).
	deposit("Malyshev", "Sber", 364562663, 10000).
	
	all_owns(Surname, Name) :- owner(Surname, car(Name, _, _)).
	all_owns(Surname, Name) :- owner(Surname, building(Name, _)).
	all_owns(Surname, Name) :- owner(Surname, region(Name, _)).
	all_owns(Surname, Name) :- owner(Surname, water_transport(Name, _, _)).
	
	all_owns_cost(Surname, Name, Cost) :- owner(Surname, car(Name, _, Cost)).
	all_owns_cost(Surname, Name, Cost) :- owner(Surname, building(Name, Cost)).
	all_owns_cost(Surname, Name, Cost) :- owner(Surname, region(Name, Cost)).
	all_owns_cost(Surname, Name, Cost) :- owner(Surname, water_transport(Name, _, Cost)).
	
	all_owns_cost_type(Surname, car, Cost) :- owner(Surname, car(_, _, Cost)), !.
	all_owns_cost_type(Surname, building, Cost) :- owner(Surname, building(_, Cost)), !.
	all_owns_cost_type(Surname, region, Cost) :- owner(Surname, region(_, Cost)), !.
	all_owns_cost_type(Surname, water_transport, Cost) :- owner(Surname, water_transport(_, _, Cost)), !.
	all_owns_cost_type(_, _, 0).

	total_cost(Surname, Total) :- 
								all_owns_cost_type(Surname, car, Cost1),
								all_owns_cost_type(Surname, building, Cost2),
								all_owns_cost_type(Surname, region, Cost3),
								all_owns_cost_type(Surname, water_transport, Cost4),
								Total = Cost1 + Cost2 + Cost3 + Cost4.

goal
	all_owns("Malyshev", Name).
	all_owns_cost("Shatskiy", Name, Cost).
	total_cost("Gribochkov", Total).
\end{lstlisting}

\section*{Порядок работы}

\begin{table}[H]
	\begin{center}
		\begin{tabular}{|p{1 cm}|p{10 cm}|p{8 cm}|}
			\hline
			№ шага & Сравниваемые термы; результат; подстановка, если есть & Дальнейшие действия: прямой ход или откат (к чему приводит?) \\
			\hline 
			
		\end{tabular}
	\end{center}
\end{table} 

\chapter*{Лабораторная работа №14}

\section*{Постановка задачи}

Создать базу знаний: <<ПРЕДКИ>>, позволяющую наиболее эффективным способом (за меньшее количество шагов, что обеспечивается меньшим количеством предложений БЗ – правил), и используя разные варианты (примеры) одного вопроса, определить (указать: какой вопрос для какого варианта):

\begin{enumerate}
	\item По имени субъекта определить всех его бабушек (предки 2-го колена);
	\item По имени субъекта определить всех его дедушек (предки 2-го колена);
	\item По имени субъекта определить всех его бабушек и дедушек (предки 2-го колена);
	\item По имени субъекта определить его бабушку по материнской линии (предки 2-го колена);
	\item По имени субъекта определить его бабушку и дедушку по материнской линии (предки 2-го колена).
\end{enumerate}

Минимизировать количество правил и количество вариантов вопросов. Использовать конъюнктивные правила и простой вопрос.

Для одного из вариантов ВОПРОСА и конкретной БЗ составить таблицу, отражающую конкретный порядок работы системы, с объяснениями:

\begin{itemize}
	\item очередная проблема на каждом шаге и метод ее решения,
	\item каково новое текущее состояние резольвенты, как получено,
	\item какие дальнейшие действия? (запускается ли алгоритм унификации? Каких термов? Почему этих?),
	\item вывод по результатам очередного шага и дальнейшие действия.
\end{itemize}

Так как резольвента хранится в виде стека, то состояние резольвенты требуется отображать в столбик: вершина – сверху! Новый шаг надо начинать с нового состояния резольвенты!

\newpage
\section*{Решение}

\begin{lstlisting}
domains
	name, sex = string
	human = human(name, sex)

predicates
	is_parent(human, human)
	is_grandparent(human, sex, name)

clauses
	is_grandparent(human(GName, GSex), PSex, Name) :- 
			is_parent(human(GName, GSex), human(TmpName, PSex)), is_parent(human(TmpName, _), human(Name, _)).
	
	is_parent(human("Elena", f), human("Ivan", m)).
	is_parent(human("Alexey", m), human("Ivan", m)).
	is_parent(human("Nina", f), human("Alexey", m)).
	is_parent(human("Anatoly", m), human("Alexey", m)).
	is_parent(human("Vera", f), human("Elena", f)).
	is_parent(human("Evgeny", m), human("Elena", f)).

goal
	is_grandparent(human(GName, f), _, "Ivan").
	is_grandparent(human(GName, m), _, "Ivan").
	is_grandparent(human(GName, _), _, "Ivan").
	is_grandparent(human(GName, f), f, "Ivan").
	is_grandparent(human(GName, _), f, "Ivan").
\end{lstlisting}

\section*{Порядок работы}

\begin{table}[H]
	\begin{center}
		\begin{tabular}{|p{0.92 cm}|p{5 cm}|p{6.5 cm}|p{6 cm}|}
			\hline
			№ шага & Состояние резольвенты, и вывод: дальнейшие действия (почему?) & Для каких термов запускается алгоритм унификации: Т1=Т2 и каков результат (и подстановка) & Дальнейшие действия: прямой ход или откат (почему и к чему приводит?) \\
			\hline 
			
		\end{tabular}
	\end{center}
\end{table} 

\end{document}