\documentclass[12pt]{report}
\usepackage[utf8]{inputenc}
\usepackage[english, russian]{babel}
\usepackage{listings}
\usepackage{graphicx}
\usepackage{float}
\graphicspath{{imgs/}}
\usepackage{amsmath,amsfonts,amssymb,amsthm,mathtools} 
\usepackage{pgfplots}
\usepackage{filecontents}
\usepackage{indentfirst}
\usepackage{eucal}
\usepackage{enumitem}
\frenchspacing

\usepackage{indentfirst} % Красная строка

\usetikzlibrary{datavisualization}
\usetikzlibrary{datavisualization.formats.functions}

\usepackage{amsmath}
\usepackage{fixltx2e}
\usepackage{caption}


\definecolor{bluekeywords}{rgb}{0,0,1}
\definecolor{greencomments}{rgb}{0,0.5,0}
\definecolor{redstrings}{rgb}{0.64,0.08,0.08}
\definecolor{xmlcomments}{rgb}{0.5,0.5,0.5}
\definecolor{types}{rgb}{0.17,0.57,0.68}

\usepackage{listings}
\lstset{language=[Sharp]C,
	captionpos=t,
	numbers=left, %Nummerierung
	numberstyle=\small, % kleine Zeilennummern
	frame=single, % Oberhalb und unterhalb des Listings ist eine Linie
	stepnumber=1,                   
	numbersep=5pt,                
	showspaces=false,
	tabsize=2,
	showtabs=false,
	breaklines=true,
	showstringspaces=false,
	breakatwhitespace=true,
	escapeinside={(*@}{@*)},
	commentstyle=\color{greencomments},
	morekeywords={partial, var, value, get, set},
	keywordstyle=\color{bluekeywords},
	stringstyle=\color{redstrings},
	basicstyle=\ttfamily\small,
}

\usepackage[left=2cm,right=2cm, top=1cm,bottom=2cm,bindingoffset=0cm]{geometry}
% Для измененных титулов глав:
\usepackage{titlesec, blindtext, color} % подключаем нужные пакеты
\definecolor{gray75}{gray}{0.75} % определяем цвет
\newcommand{\hsp}{\hspace{20pt}} % длина линии в 20pt
% titleformat определяет стиль
\titleformat{\chapter}[hang]{\Huge\bfseries}{\thechapter\hsp\textcolor{gray75}{|}\hsp}{0pt}{\Huge\bfseries}

\usepackage{array}
\newcommand{\head}[2]{\multicolumn{1}{>{\centering\arraybackslash}p{#1}}{#2}}

% plot
\usepackage{pgfplots}
\usepackage{filecontents}
\usetikzlibrary{datavisualization}
\usetikzlibrary{datavisualization.formats.functions}

\begin{document}
	%\def\chaptername{} % убирает "Глава"
	\thispagestyle{empty}
	\begin{titlepage}
		\noindent \begin{minipage}{0.15\textwidth}
			\includegraphics[width=\linewidth]{b_logo}
		\end{minipage}
		\noindent\begin{minipage}{0.9\textwidth}\centering
			\textbf{Министерство науки и высшего образования Российской Федерации}\\
			\textbf{Федеральное государственное бюджетное образовательное учреждение высшего образования}\\
			\textbf{~~~«Московский государственный технический университет имени Н.Э.~Баумана}\\
			\textbf{(национальный исследовательский университет)»}\\
			\textbf{(МГТУ им. Н.Э.~Баумана)}
		\end{minipage}
		
		\noindent\rule{18cm}{3pt}
		\newline\newline
		\noindent ФАКУЛЬТЕТ $\underline{\text{«Информатика и системы управления»}}$ \newline\newline
		\noindent КАФЕДРА $\underline{\text{«Программное обеспечение ЭВМ и информационные технологии»}}$\newline\newline\newline\newline\newline\newline\newline\newline\newline\newline\newline
		
		
		\begin{center}
			\noindent\begin{minipage}{1.3\textwidth}\centering
				\Large\textbf{  Отчет по лабораторной работе №1}\newline
				\textbf{по дисциплине \newline "Математическая статистика"}\newline\newline
			\end{minipage}
		\end{center}
		
		\noindent\textbf{Тема} $\underline{\text{Гистограмма и эмпирическая функция распределения}}$\newline\newline
		\noindent\textbf{Студент} $\underline{\text{Малышев И. А.}}$\newline\newline
		\noindent\textbf{Группа} $\underline{\text{ИУ7-61Б}}$\newline\newline
		\noindent\textbf{Оценка (баллы)} $\underline{\text{~~~~~~~~~~~~~~~~~~~~~~~~~~~}}$\newline\newline
		\noindent\textbf{Преподаватель: } $\underline{\text{Власов П. А.}}$\newline\newline\newline
		
		\begin{center}
			\vfill
			Москва~---~\the\year
			~г.
		\end{center}
	\end{titlepage}
	
\setcounter{page}{2}

\chapter*{Задание}

\section*{Цель работы}
Построение гистограммы и эмпирической функции распределения.

\section*{Постановка задачи}

\begin{enumerate}
	\item Для выборки объёма $n$ из генеральной совокупности $X$ реализовать в виде программы на ЭВМ
	\begin{enumerate}
		\item вычисление максимального значения $M_{\max}$ и минимального значения $M_{\min}$;
		\item размаха $R$ выборки;
		\item вычисление оценок $\hat\mu$ и $S^2$ математического ожидания $MX$ и дисперсии $DX$;
		\item группировку значений выборки в $m = [\log_2 n] + 2$ интервала;
		\item построение на одной координатной плоскости гистограммы и графика функции плотности распределения вероятностей нормальной случайной величины с математическим ожиданием $\hat{\mu}$ и дисперсией $S^2$;
		\item построение на другой координатной плоскости графика эмпирической функции распределения и функции распределения нормальной случайной величины с математическим ожиданием $\hat{\mu}$ и дисперсией $S^2$.
	\end{enumerate}
	\item Провести вычисления и построить графики для выборки из индивидуального варианта.
\end{enumerate}

\chapter*{Теоретические сведения}

\section*{Формулы для вычисления величин}

\subsection*{Минимальное и максимальное значения выборки}
\begin{equation}
	\begin{aligned}
		M_{\max} = X_{(n)}\\
		M_{\min} = X_{(1)}
	\end{aligned}
\end{equation}

\subsection*{Размах выборки}
\begin{equation}
	R = M_{\max} - M_{\min}.
\end{equation}

\subsection*{Оценки выборочного математического ожидания и исправленной выборочной дисперсии}
\begin{equation}
	\begin{aligned}
		\hat\mu(\vec X_n) &= \frac 1n \sum_{i=1}^n X_i\\
		S^2(\vec X_n) &= \frac 1{n-1} \sum_{i=1}^n (X_i-\overline X_n)^2
	\end{aligned}
\end{equation}

\section*{Определение эмпирической плотности и гистограммы}

Пусть $\vec x$ -- выборка из генеральной совокупности $X$. Если объем $n$ этой выборки велик, то значения $x_i$ группируют в интервальный статистический ряд. Для этого отрезок $J = [x_{(1)}, x_{(n)}]$ делят на $m$ равновеликих частей:

\begin{equation*}
	J_i = [x_{(1)} + (i - 1) \cdot \Delta, x_{(1)} + i \cdot \Delta), i = \overline{1; m - 1}
\end{equation*}

\begin{equation*}
	J_{m} = [x_{(1)} + (m - 1) \cdot \Delta, x_{(n)}]
\end{equation*}

\begin{equation*}
	\Delta = \frac{|J|}{m} = \frac{x_{(n)} - x_{(1)}}{m}
\end{equation*}

Интервальным статистическим рядом называют таблицу:

\begin{table}[htb]
	\centering
	\begin{tabular}{|c|c|c|c|c|}
		\hline
		$J_1$ & ... & $J_i$ & ... & $J_m$ \\
		\hline
		$n_1$ & ... & $n_i$ & ... & $n_m$ \\
		\hline
	\end{tabular}
\end{table}

где $n_i$ -- количество элементов выборки $\vec x$, которые $\in J_i$.

Обычно выборку разбивают на $m=[\log_2n]+2$ интервалов, где $n$ -- размер выборки.

Гистограмма -- это график эмпирической плотности. 

\textit{Эмпирической плотностью}, отвечающей выборке $\vec x$, называют функцию:
\begin{equation}
	\hat f(x) =
	\begin{cases}
		\frac{n_i}{n \Delta}, x \in J_i, i = \overline{1; m} \\
		0, \text{иначе} \\
	\end{cases}
\end{equation}

где $J_i$ -- полуинтервал статистического ряда, $n_i$ -- количество элементов выборки, входящих в полуинтервал, $n$ -- количество элементов выборки.

\section*{Определение эмпирической функции распределения}

Пусть $\vec x = (x_1, ..., x_n)$ -- выборка из генеральной совокупности $X$. Обозначим $n(x, \vec x)$ -- число элементов вектора $\vec x$, которые имеют значения меньше $x$.

\textit{Эмпирической функцией распределения} называют функцию $F_n: \mathbb{R} \to \mathbb{R}$, определенную как: 

\begin{equation}
	F_n(x) = \frac{n(x, \vec x)}{n}
\end{equation}

\chapter*{Результаты работы программы}

\section*{Текст программы}
\begin{lstlisting}[language=Matlab]

\end{lstlisting}

\section*{Результаты расчётов}

\begin{itemize}
	\item $M_{\max} = 12.4100$
	\item $M_{\min} = 6.8100$
	\item $R = 5.6000$
	\item $\hat\mu(\vec x_n) = 9.4872$
	\item $S^2(\vec x_n) = 1.2173$
	\item $m = 8$
\end{itemize}

\end{document}